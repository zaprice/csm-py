\documentclass[a4paper]{amsproc}

\usepackage{amsmath}
\usepackage{amssymb}
\usepackage{amsthm}
%%\usepackage[margin=.7in]{geometry}
%%\usepackage[hyphens]{url} \urlstyle{same}
\usepackage{tikz}
\usepackage{hyperref}
\usepackage{float}
\usepackage[labelsep=none]{caption}

%%\setlength{\textwidth}{28cc} \setlength{\textheight}{42cc}

\theoremstyle{plain}
\newtheorem{thm}{Theorem}[section]
\newtheoremstyle{named}{}{}{\itshape}{}{\bfseries}{.}{.5em}{\thmnote{#3 }#1}
\theoremstyle{named}
\newtheorem*{namedtheorem}{Theorem}
\theoremstyle{definition}
\newtheorem{definition}{Definition}
\newtheorem{corollary}{Corollary}
%%\setlength{\parindent}{0pt}

\newcommand{\RR} {\color{red} R}
\newcommand{\BB} {\color{blue} B}
\newcommand{\GG} {\color{green} G}
\newcommand{\YY} {\color{yellow} Y}
\newcommand{\N} {\mathbb{N}}
\newcommand{\Q} {\mathbb{Q}}

\title{Statistically optimal configurations for cyber-security models}
\author[Price]{Zachary Price}
\thanks{Department of Mathematics, George Mason University. Advised by Dr. Geir Agnarsson. Supported by the GMU Mathematics Industrial Immersion Program, sponsored by GMU Provost PhD award}
\email{zprice3@masonlive.gmu.edu}
\date{\today}

\begin{document}


\tikzstyle{vertex} = [circle,draw,fill=white!20]

\vspace{18mm} \setcounter{page}{1} \thispagestyle{empty}

\begin{abstract}
  A cyber-security model is an ordered three-tuple $M = (T, C, P)$ where $T$ is a tree and $C$ and $P$ are multisets of costs and prizes, respectively, with cardinality $|T|-1$.
  Building off of previous work, we formulate a notion of statistical optimality for given $(T, C, P)$ and explore computationally what such configurations would look like.
\end{abstract}

\maketitle

\section{Background}

The study of abstract cyber-security models has been pioneered in previous papers \cite{agk1} and \cite{agk2}.
Here we state a few important definitions and results.

\begin{definition}
A \emph{cyber-security model} (CSM) $M$ is given by a three-tuple $M= (T, C, P)$, where $T$ is a directed tree rooted at $r$ having $n\in\N$ non-root vertices, $C$ is a multiset of penetration costs $c_1,\ldots,c_n\in\Q_+$, and $P$ is a multiset of target-acquisition-values (prizes for short) $p_1,\ldots,p_n\in\Q_+$.
\end{definition}

\begin{definition}
A \emph{security system} (SS) with respect to a cyber-security model $M=(T,C,P)$ is given by two bijections $c:E(T)\rightarrow C$ and $p:V(T)\setminus\{r\}\rightarrow P$.
We denote the security system by $(T,c,p)$.
A \emph{system attack} (SA) in a security system $(T,c,p)$ is given by a subtree $\tau$ of $T$ that contains the root $r$.

The \emph{cost} of a system attack $\tau$ with respect to $(T,c,p)$ is defined by
$$cst(\tau,c,p)=\sum_{e\in E(\tau)}c(e)$$

The \emph{prize} of a system attack $\tau$ with respect to $(T,c,p)$ is defined by
$$pr(\tau,c,p)=\sum_{u\in V(\tau)} p(u)$$

For a given \emph{budget} $B\in\Q_+$, the maximum prize $pr^*(B,c,p)$ w.r.t. $B$ is
$$pr^*(B,c,p) = \max\{pr(\tau,c,p) : cst(\tau,c,p)\leq B\}$$
\end{definition}

\begin{definition} Let $M=(T,C,P)$ be a given CSM. For a budget $B\in\Q_+$, a SS $(T,c,p)$ is \emph{optimal w.r.t. $B$} if there is no other SS $(T,c',p')$ for $M$ such that $pr^*(B,c',p')<pr^*(B,c,p)$. $(T,c,p)$ is called \emph{optimal} if it is optimal w.r.t. any budget $B\in\Q_+$.
\end{definition}

In \cite{agk2}, various facts are proved about optimal security systems and certain simplified models of CSMs.
In particular, a classification theorem is proved showing that only certain classes of trees always admit an optimal SS.

\section{Methods}

First, we generalize the previous definition of \emph{optimal}.
We will be dealing only with budgets $B\in\N_+$ for simplicity.

\begin{definition}Let $M=(T,C,P)$ be a given CSM, with a particular SS $(T,c,p)$.
  The \emph{payoff sum} is
  $$ps(T,c,p) = \sum_{B=1}^{B_{\max}} pr^*(B,c,p)$$
  where $B_{\max}$ is the budget required to obtain all prizes in $P$.
\end{definition}

The payoff sum can be conceptualized as the area under the \emph{prize-budget curve}, which relates each budget to the maximum prize obtainable for that budget in a given SS.
This allows us to compare the performance, on average, of different configurations:
one SS may be superior to another SS if it gives lower prizes at more budgets.

\begin{definition}Let $M=(T,C,P)$ be a given CSM. A SS $(T,c,p)$ is \emph{statistically optimal} if $ps(T,c,p)\leq ps(T,c',p')$ for any other SS $(T,c',p')$.
\end{definition}

\begin{corollary}
  If $(T,c,p)$ is optimal for $M$, then it is statistically optimal.
\end{corollary}
\begin{proof}
  For any other SS $(T,c',p')$, $pr^*(B,c,p) \leq pr^*(B,c',p')$ for every budget $B$.
  Thus, $\sum_{B=1}^{B_{\max}} pr^*(B,c,p) \leq \sum_{B=1}^{B_{\max}} pr^*(B,c',p')$.
\end{proof}

We know from \cite{agk2} that not all trees have optimal configurations; in fact, almost all sufficiently large trees have no optimal SS.
Every tree has a statistically optimal SS, however.
Thus, from the corollary, statistical optimality is a generalization of true optimality for security systems.


\section{Results}

The goal is to find an efficient procedure that takes a CSM and gives a labeling that is statistically optimal.
This would be a significant advancement in the study of CSMs from \cite{agk2}: for any given network, one could quickly decide how best to lay out layered-security defenses.

Given a rooted tree $T$ and cost/prize sets $C,P$, we used computational methods to search for all $c,p$ that give statistically optimal security systems.
We try each labeling $C\rightarrow E(T)$, $P\rightarrow V(T)\setminus\{r\}$, checking every subtree of $T$ for optimal tree attacks at each budget, and then compute the payoff sum.
The output is $c,p$ from a statistically optimal SS.
A Python implementation of this procedure can be found on Github at \url{https://github.com/zaprice/csm-py}

This gives results for small enough trees ($|V|<7$) but could be scaled up with clever branch pruning and more efficient programming.
Our results show that statistically optimal security systems are \emph{not} unique.
A given $(T, C, P)$ may have multiple, fairly different configurations with the same payoff sum; different security systems will over- or under-perform each other at certain budgets, but these differences cancel out in aggregate.

Some examples of the output follow.
The root is labeled $0$, while the other vertices are labeled with their prizes.
Edges are labeled with their respective costs.
Figures 1 and 2 show two different statistically optimal security systems for the same tree.

\begin{figure}[H]
\includegraphics[scale=0.75]{all_best_2_5.png}
\caption{}
\label{fig:M0}
\end{figure}x

\begin{figure}[H]
\includegraphics[scale=0.75]{all_best_2_6.png}
\caption{}
\label{fig:M1}
\end{figure}

\begin{figure}[H]
\includegraphics[scale=0.75]{7_tree_7.png}
\caption{}
\label{fig:M2}
\end{figure}




\section{Future Work}

We hope to use these concrete examples to generate a conjecture about a labeling algorithm, after which we will try to prove that it produces only statistically optimal labelings.
The code used to generate statistically optimal configurations could also be used to explore other generalizations of the CSM paradigm, including non-linear prize sets.

\begin{thebibliography}{9}

\bibitem{agk1}
  Geir Agnarsson, Raymond Greenlaw, Sanpawat Kantabutra.
  The Complexity of Cyber Attacks in a New Layered-Security Model and the Maximum-Weight, Rooted-Subtree Problem,
  \emph{Cybernetica}, to appear.

\bibitem{agk2}
  Geir Agnarsson, Raymond Greenlaw, Sanpawat Kantabutra.
  The structure and topology of rooted weighted trees modeling layered cyber-security systems.

\end{thebibliography}

\end{document}
